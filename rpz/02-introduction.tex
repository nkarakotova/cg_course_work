\chapter*{ВВЕДЕНИЕ}
\addcontentsline{toc}{chapter}{ВВЕДЕНИЕ}

В XXI веке с возрастанием компьютерных возможностей и с расширением применения компьютерных технологий в различных сферах растёт так же необходимость в графическом моделировании. Компьютерная графика используется практически во всех научных и инженерных дисциплинах для наглядности восприятия и передачи информации, а так же в игровой индустрии и кинематографе. \cite{tomsk_2012}

Построение реалистичного изображения мыльных пузырей может понадобиться для моделирования их физической природы.

Целью работы является разработка программного обеспечения для создания реалистичного изображения мыльных пузырей.

Для достижения данной цели требуется решить следующие задачи:
\begin{enumerate}[label=\arabic*)]
    \item описать физическую модель мыльных пузырей;
    \item проанализировать и выбрать модели представления объектов;
    \item проанализировать и выбрать алгоритмы решения основных задач компьютерной графики: удаления невидимых линий и поверхностей, учёта теней и освещения;
    \item спроектировать программное обеспечение;
    \item выбрать средства реализации и реализовать спроектированное программное обеспечение;
    \item обеспечить возможность тестирования, создать наборы тестов и продемонстрировать работоспособность программы;
    \item исследовать характеристики разработанного программного обеспечения.
\end{enumerate}